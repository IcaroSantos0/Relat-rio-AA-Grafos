\documentclass{article}
\usepackage[utf8]{inputenc}
\usepackage{color}
\usepackage{ragged2e}

\title{Relatório para a Atividade Acadêmica de Grafos}
\author{Ícaro.S.Barcelos}
\date{Novembro 2019}

\begin{document}

\maketitle

\section{Resumo}

\setlength{\parindent}{5ex} 

O relatório a seguir tem como objetivo exibir um algoritmo para o reconhecimento de grafos cordais bipartidos. Esta classe é aplicada em diversos problemas relacionados a matrizes não-simétricas, eliminação Gaussiana de matrizes, programação inteira e análise de matrizes.

\pagebreak
\section{Introdução}
\setlength{\parindent}{5ex}
Um grafo é bipartido cordal se é bipartido e cada um de seus ciclos de tamanho maior ou igual a 6 tem uma corda. Como um grafo bipartido só possui ciclos induzidos de tamanho par, um grafo bipartido cordal só pode possuir ciclos induzidos de tamanho 4.\par
Note que em geral os grafos bipartidos cordais não são cordais. Por exemplo, o C4 é bipartido cordal mas não é cordal. Naturalmente, também existem grafos cordais que não são bipartidos cordais como o C3, por exemplo. Portanto, um grafo bipartido cordal não é necessariamente um grafo bipartido e cordal, como poderia ser imaginado baseando-se no nome dado à classe.
\end{document}